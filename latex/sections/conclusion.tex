\thispagestyle{plain}
\noindent To summarize our findings, it is clear that the choice of the best regression method depends on the data set used. In scenarios where the data is noise-free, OLS proves to be the most effective regression method. But noise-free data is usually not the case. What we found was that the best regression method for the Franke function with noise given by the normal distribution $\mathcal{N}(0, 0.1)$, was Ridge for higher polynomial degrees, this is due to OLS over-fitting the data to the noise for higher polynomial degrees, while Ridge and LASSO tries to minimise the variance in $\beta$ values to avoid this problem. 
Even though LASSO aims to minimize variance in $\beta$ values, it exhibited the worst performance among the three regression methods. However, it remains uncertain whether Lasso could outperform Ridge and OLS in handling noise due to limitations in iterations. The decision to restrict iterations was influenced by run time constraints, leaving the true capabilities of Lasso in this scenario unknown.
\noindent For our terrain data we surprisingly found that OLS gave the model with the lowest MSE. From what we can see in figure \eqref{OLS 3D figure terrain data} the terrain data looks a little noise, but it may be due to the scaling of the data set that makes OLS the best regression method for the terrain data. 

\noindent The improvement potential for this project is high, due to some grope problems we got started on the project late together as one group. Therefor the structure on GitHub and the cods are quite messy. This is something we will work on improving for the next project. 
\documentclass[reprint,english,notitlepage]{revtex4-1}  % 
\usepackage[utf8]{inputenc}
\usepackage[english]{babel}
\usepackage{physics,amssymb} 
\usepackage{graphicx}        
\usepackage{xcolor}          
\usepackage{hyperref}        
\usepackage{tikz}             
\usepackage{listings}       
\usepackage{subfigure}  
\usepackage{here}
\hypersetup{ % this is just my personal choice, feel free to change things
    colorlinks,
    linkcolor={red!50!black},
    citecolor={blue!50!black},
    urlcolor={blue!80!black}}

%% Defines the style of the programming listing
%% This is actually my personal template, go ahead and change stuff if you want
\lstset{ %
	inputpath=,
	backgroundcolor=\color{white!88!black},
	basicstyle={\ttfamily\scriptsize},
	commentstyle=\color{magenta},
	language=Python,
	morekeywords={True,False},
	tabsize=4,
	stringstyle=\color{green!55!black},
	frame=single,
	keywordstyle=\color{blue},
	showstringspaces=false,
	columns=fullflexible,
	keepspaces=true}


%% USEFUL LINKS:
%%
%%   UiO LaTeX guides:        https://www.mn.uio.no/ifi/tjenester/it/hjelp/latex/ 
%%   mathematics:             https://en.wikibooks.org/wiki/LaTeX/Mathematics

%%   PHYSICS !                https://mirror.hmc.edu/ctan/macros/latex/contrib/physics/physics.pdf

%%   the basics of Tikz:       https://en.wikibooks.org/wiki/LaTeX/PGF/TikZ
%%   all the colors!:          https://en.wikibooks.org/wiki/LaTeX/Colors
%%   how to draw tables:       https://en.wikibooks.org/wiki/LaTeX/Tables
%%   code listing styles:      https://en.wikibooks.org/wiki/LaTeX/Source_Code_Listings
%%   \includegraphics          https://en.wikibooks.org/wiki/LaTeX/Importing_Graphics
%%   learn more about figures  https://en.wikibooks.org/wiki/LaTeX/Floats,_Figures_and_Captions
%%   automagic bibliography:   https://en.wikibooks.org/wiki/LaTeX/Bibliography_Management  (this one is kinda difficult the first time)
%%   REVTeX Guide:             http://www.physics.csbsju.edu/370/papers/Journal_Style_Manuals/auguide4-1.pdf
%%
%%   (this document is of class "revtex4-1", the REVTeX Guide explains how the class works)


%% CREATING THE .pdf FILE USING LINUX IN THE TERMINAL
%% 
%% [terminal]$ pdflatex template.tex
%%
%% Run the command twice, always.
%% If you want to use \footnote, you need to run these commands (IN THIS SPECIFIC ORDER)
%% 
%% [terminal]$ pdflatex template.tex
%% [terminal]$ bibtex template
%% [terminal]$ pdflatex template.tex
%% [terminal]$ pdflatex template.tex
%%
%% Don't ask me why, I don't know.

\begin{document}
\title{Project 1 FYS-STK4155}   % self-explanatory
\author{}               % self-explanatory
\date{\today}                             % self-explanatory
\noaffiliation   

% ignore this
\begin{abstract}                          % marks the 


\end{abstract}                            % marks the end of the abstract
\maketitle                                % creates the title, author, date & abstract


% the fundamental components of scientific reports:
\section{Introduction}

\thispagestyle{plain}
Machine learning is a powerfull tool usfull in many fields of reasarche.

- spatial planning problems
- flood
- avalance

Write a paragraph about why it is useful with machine learning, in this setting.

\noindent The aim of this report is to study three different regression methods, ordinary least squares (OLS), Ridge and LASSO and see how these method compare to each other when applied to different data sets. First we are going to look at the Franke function. When plotted between 0 and 1 this function looks like a mountain and a valley, which is a perfect starting point when we later want to apply these methods on digital terrain data taken from \url{https://earthexplorer.usgs.gov/}.

\section{Theory}   % (optional)
\thispagestyle{plain}

\subsection{Regression methods}
\subsubsection{Ordinary least squares (OLS)}
\subsubsection{Ridge}
\subsubsection{LASSO}
\subsection{MSE}


\subsection{Resampling techniques}

\noindent The main restriction in machine learning is 
the amount of data points available to create the model out of. It may be the case where one have
done a costfull and time consuming experiments and are left with a small number of data. 
It is therfull extremely useful to have methods where one can reuse the data multiple times
thereby creating a relatively large dataset from the small number of datapoints. In this report
we are going to use two different methods, the first is called bootstrap and the second one is cross validation.

\subsubsection{Bootstrap} 
\noindent The bootstrap method is a resampling procedure that uses data from one 
sample to generate a sampling distribution by repeatedly taking random 
samples from the known sample, with replacement\cite{PSU} This means that if we have a data set D with n data points.
The elements in this data set can be representatet in the following way:
\begin{align}
    D = {d_1, d_2, d_3, d_4, \dots, d_n}
\end{align}
Then by apllying the bootstarp method on this data set one possible output $D^{*}$ can be:
\begin{align}
    D^{*} = {d_3, d_n, d_4, d_4 , \dots, d_2}
\end{align}
From this example we see that one observation can appear multiple times in the new dataset.
We can take this method a step further and create "new" data points by extracting multible data points from 
the dataset and take the mean of all thise values:
\begin{align}
    d_{new} = \frac{1}{k}(d_1, \dots d_k)
\end{align}
This gives us a method of producing lots of "new" data-set from 
limited data points to train our model with. For each of this data-sets
the mean and standard deviation can be calculated to evaluate the model statistically. \cite{MLM}


One huge advatage of using the boostrap method is that the data can be split in to 
test and train before shuffeling the data, this means that the test data can bee kept 
entirely separate from the creation of the model. When we den test the model it will
be on a dataset that has nothing to do with crating the model and will therfore show
how god the model represent real data. 
%\newline
\textbf{Write something about large numbers law, also disadvantages and advantage}
%\newline


\subsubsection{Cross validation}
\noindent Cross validation is another method of creating "new" datasets from the original data. This method 
wrks by splitting the data in k-folds
\subsection{Bias-variance trade-off}

\section{Method}
\thispagestyle{plain}

\noindent In the first part of this project a function called Franke function
 was used as the data analysed. The Franke function is given by the following 
 equation:
\begin{align*}
    f(x,y) &= \frac{3}{4} \, exp\left(- \frac{(9x-2)^2}{4} - \frac{(9y-2)^2}{4}\right) \\
    &+ \frac{3}{4}\, exp\left( - \frac{(9x +1)^2}{49} - \frac{(9y+1)}{10}\right) \\
    &+ \frac{1}{2}\, exp\left( -\frac{(9x-7)^2}{4} - \frac{(9y-3)^2}{4}\right) \\
    &- \frac{1}{5} exp \left( - (9x -4)^2 - (9y-7)^2\right)
\end{align*}
%A graphic representation is shown in figure \eqref{Franke function}
\begin{figure}[H]
	\centering
	\includegraphics[width=0.5\textwidth]{Figure_1.png}
	\caption{\centering A plot of the Franke function }
	\label{Franke function}
\end{figure}
\noindent
The amount of data created of the franke function was a $20 x 20$ matrix where $20\%$
of the data set was set aside as the testing dataset while the ramaining $80\%$ was used for training. The noise 
was reduced from $\mathcal{N}(0,1)$ as proposed in the project description, to $\mathcal{0,0.1}$ for better results.
First this data was fitted with the OLS method were varying degrees of polynomials was used to create the design matrix.
Since the design matrix in this case was noninvertible, singular value decomposition was used to create 
the $\beta$-values needed to create a model. After the regression was done, the mean square error and the
R2 score was calculated for both the testing and training datasets. The different coefficients, or so called $\beta$
values was plotted for polynomial degrees in range $0-5$. The result for this analysis of the franke function
is shown in figures \eqref{MSE and R2 OLS}, \eqref{MSE and R2 OLS noise} and ... 


\noindent Next the bootstrap method was implementet with .... iterations to se how this 
affected the result of the OLS regression analysis. This method was implemented after the dataset
was split into a training and testing set, so that the testing dataset didn't have
anything to do with the creation of the diffrent $\beta$ values. 

\noindent The last thing done with OLS on the franke function was cross validation.. 

\textbf{Need to rewrite the method section about cross validation}
In the pursuit of assessing the robustness and reliability of Ridge regression models, we employed cross-validation, specifically scrutinizing the impact of varying regularization parameters, denoted as \(\lambda\). The function \texttt{k\_fold}, was developed to execute the k-fold cross-validation technique, accepting the dataset and an integer \(k\) as arguments, and subsequently partitioning the data into \(k\) randomized subsets (folds). It returns \(k\) pairs of training and test indices, each representing a distinct division of the data, enabling model evaluation across varied data scenarios.

For each \(\lambda\) value in a logarithmically spaced array of \(\lambda\) values, denoted \texttt{lambdas}, the Ridge regression model was trained and validated \(k\) times - once per fold. Specifically, for every tuple \((\text{train\_indices}, \text{test\_indices})\) produced by \texttt{k\_fold(data, k)}, the data was divided into training and test sets \((x_{\text{train}}, y_{\text{train}}, x_{\text{test}}, y_{\text{test}})\).
Subsequently, the \texttt{design\_matrix} function generated polynomial feature matrices \(X_{\text{train}}\) and \(X_{\text{test}}\) from the \(x\) and \(y\) values, using a specified polynomial degree. 

The Ridge model, instantiated with the current \(\lambda\), was then fitted with \(X_{\text{train}}\) and \(y_{\text{train}}\), and predictions \(y_{\text{pred}}\) were made using \(X_{\text{test}}\). The Mean Squared Error (MSE) between the predictions and actual test values \(y_{\text{test}}\) was computed and stored in a scores array.
The MSE values were averaged per \(\lambda\), providing an unbiased performance metric, and facilitating the analysis and visualization of how distinct regularization parameters influenced model performance.


\noindent Next Ridge and LASSO regression was used on the Franke function,
to see if these methods have a better fit than what was obtained with OLS.
Diffrent values for $\lambda$ was used to obtain the best fit as possible for each
polynomial degree. For Ridge equation \eqref{beta ridge} was used to calculate
the cofficents and for LASSO ..




In the last part of this project real terrain data was analysed. Due to the massiv size 
of the terrain data only the first $500 \cdot 500$ matrix as shown in figure \eqref{terrain data} was used to avoid ram problems. 
\begin{figure}[h]
	\centering
	\includegraphics[width=0.5\textwidth]{Figure_11.png}
	\caption{A plot of the data matrix used in the analysis of the terrain data in the second part of project 1.}
	\label{terrain data}
\end{figure}
The OLS regression was applyed to creat a modell of the dataset 



\section{Results}
\thispagestyle{plain}
\subsection{Result for Franke function}
\noindent Our initial step involved the application of OLS to the Franke function without noise.	
This analysis was conducted without employing any resampling techniques, 
and our dataset conisted of $20 \cdot 20$ data points. The results for the MSE and R2 score is shown in 
figure \eqref{MSE and R2 OLS}

\begin{figure}[H]
	\centering
	\includegraphics[width=0.5\textwidth]{Figure_3.png}
	\caption{Plot showing the MSE and R2 score for the franke function without noise. The regression method used was OLS}
	\label{MSE and R2 OLS}
\end{figure}
\noindent The next step was to include noise given by the normal distribution $\mathcal{N}(0,0.1)$. Figure \eqref{MSE and R2 OLS noise} shows
how the MSE and R2 score changes when noise is includet in to the data set.
\begin{figure}[H]
	\centering
	\includegraphics[width=0.5\textwidth]{Figure_4.png}
	\caption{Plot showing the MSE and R2 score for the franke function with noise $\mathcal{N}(0,0.1)$. The regression method used was OLS}
	\label{MSE and R2 OLS noise}
\end{figure}
Lastly we plotted the coefficients for the different orders of polynomials to see how these varies in value.
\begin{figure}[H]
	\centering
	\includegraphics[width=0.5\textwidth]{Figure_12.png}
	\caption{A plott showing the $\beta$ values for OLS for different orders of polynimials}
	\label{beta OLS}
\end{figure}



\noindent For Ridge Regression on the franke function have we plotted a heatmap to show 
how the MSE changes with diffrent $\lambda$ values and complexities. A plot of
the heatmap for the training data is shown in figure \eqref{heatmap training ridge}
, and the heatmap for the test data is shown in figure \eqref{heatmap test ridge}.
\begin{figure}[H]
	\centering
	\includegraphics[width=0.5\textwidth]{Figure_8.png}
	\caption{A heatmap of the MSE for the training data, for different $\lambda$ values and complexities. The $\lambda$ values goes from $10^{-8}$ to $10^{2}$ }
	\label{heatmap training ridge}
\end{figure}
\begin{figure}[H]
	\centering
	\includegraphics[width=0.5\textwidth]{Figure_7.png}
	\caption{A heatmap of the MSE for the testing data, for different $\lambda$ values and complexities. The $\lambda$ values goes from $10^{-5}$ to $10^{5}$}
	\label{heatmap test ridge}
\end{figure}
Lastly for Ridge regression we see what happens when $\lambda$ is put to zero.




\begin{figure}[H]
	\centering
	\includegraphics[width=0.5\textwidth]{Figure_9.png}
	\caption{A heatmap of the MSE for the training data, for different $\lambda$ values and complexities. The $\lambda$ values goes from $10^{-8}$ to $10^{2}$ }
	\label{heatmap training LASSO}
\end{figure}
\begin{figure}[h]
	\centering
	\includegraphics[width=0.5\textwidth]{Figure_10.png}
	\caption{A heatmap of the MSE for the testing data, for different $\lambda$ values and complexities. The $\lambda$ values goes from $10^{-5}$ to $10^{5}$}
	\label{heatmap test LASSO}
\end{figure}




\begin{figure*}[h]
	\centering
	\includegraphics[width=\textwidth]{Figure_2.png}
	\caption{A plot showing how model with different complexities fit the franke function when OLS regession has been used.}
	\label{OLS figure}
\end{figure*}


\begin{figure*}[h]
	\centering
	\includegraphics[width=\textwidth]{Figure_6.png}
	\caption{}
	\label{Ridge figure}
\end{figure*}

\begin{figure*}[h]
	\centering
	\includegraphics[width=\textwidth]{Figure_5.png}
	\caption{}
	\label{OLS figure terrain data}
\end{figure*}

\subsection{Result for cross validation}
\noindent The Mean Squared Error (MSE) is depicted on the $y$-axis, while the logarithm to the base 10 of the hyperparameter $\lambda$ ($\log_{10}(\lambda)$) for Ridge Regression is represented on the $x$-axis. Different lines in varying colors are plotted to showcase the MSE across different $\lambda$ values for varying numbers of folds ($k$) in the cross-validation, ranging from $k=5$ to $k=10$. 

\begin{figure*}[h!]
	\centering
	\includegraphics[width=\textwidth]{Images/Bootstrap_crossval_crossvalsklearn.png}
	\caption{}
	\label{Cross validation comparison with bootstap}
\end{figure*}

\subsection{Results for terrain data}

\section{Discussion}
\thispagestyle{plain}
\noindent It was stated in the project description that we shoud use a nois that was given by
the normal distribution $\mathcal{N}(0,1)$. What we found was that since the Franke
Function when plotten for $x$ and $y$ in the range between $0$ and $1$ the maximal 
value of the Franke function becomes around $1.4$. When plottet with noise that had values
between $0$ and $1$ the plot of the franke function becomes unrecognizable as shown in figure \eqref{franke noise 1}. 
\begin{figure}[H]
	\centering
	\includegraphics[width=0.5\textwidth]{Figure_13.png}
	\caption{Plot showing the Frankece function with noise given by the normal distrubution $\mathcal{N}(0,1)$ }
	\label{franke noise 1}
\end{figure}
\noindent But if instead the normal distribution $\mathcal{N}(0,0.1)$ is 
used, the dataset becomes much easier to work with, and the results becomes 
much prettier (as shown in figure \eqref{franke noise 0.1}) since noise has a tendency of ruining everything.
\begin{figure}[H]
	\centering
	\includegraphics[width=0.5\textwidth]{Figure_14.png}
	\caption{Plot showing the Frankece function with noise given by the normal distrubution $\mathcal{N}(0,0.1)$ }
	\label{franke noise 0.1}
\end{figure}
\noindent Now that we have gotten that out of the way, we can start actually
discussing the result of the regression analysis of the Franke function.

\subsection*{The Franke function}
\noindent We start with the analyses of the OLS implementation on the Franke
function. It is important to note that since the parameter x and y only had 
values from 0 to 1 when creating the Franke function, we did not see it necessary
to scale the data for any of the analysis of the franke function, since it in a way already is scaled. 
The first results are the OLS model of the franke function without noise. Figure \eqref{MSE and R2 OLS}
shows the MSE and R2 scores as a function of the polynomial degree used to create the design matrix.
The size of the dataset analysis was a $20 \cdot 20$ matrix, were $20\%$ was put aside as 
test data, and the rest was used for training. What we can see from the figure is that
the MSE is larger for lower polynomial degrees and starts to lower with higher values. 
At a fifth order polynomial degree the MSE approaches zero both for the test and training set.
This is as expected, since there is no noise present there will not be any posibility 
for overfitting the model to the noise. Therefore the MSE will become lower and 
lower for higher order polynomials as shown in figure \eqref{MSE and R2 OLS}. When 
it comes to the R2 score we see that it becomes closer and closer to one with higher 
order polynomials. This is due to our model becoming increasingly better with 
higher orders as shown form the MSE values.

Next we look at what happens when we introduce noise given by the normal distribution 
$\mathcal{N}(0,0.1)$. Figure \eqref{MSE and R2 OLS noise} show the MSE and the R2 score
for this case.Here we can see a classical example of overfitting the model to the noise.
We see that the MSE becomes better and better for the training set but for det test set
we see a sudden increas after fitting with a fourth order polynomial. This is due to
the amount of datapoints in the diffrent datasets, since the test dataset only have 
$20\%$ of the original data while the training set consist of the ramaining $80\%$.
If we increas the number of data points the predictet model for the test data will converge
towards that of the training model. 


\noindent From figure \eqref{MSE and R2 OLS} we see that when our data set does not include noise
the MSE for the training data gets lower with increas in complexity, while 
the R2 score gets closer and closer to 1. But when we introduce noise whe see from figure \eqref{MSE and R2 OLS noise}
that for our test data the MSE acctually increase with higher complexity. 

The dataset containing the test data consists of $20\%$ of the original
dataset. This implies that the number of points in the test data is 
significantly smaller than that in the training data. We anticipate 
that by increasing the number of data points, the MSE for the model 
created using the test data will converge towards that of the training
model.

What we can see from figure \eqref{beta OLS} is that for the higher order 
polynomials the values for the coefficients varies a lot in value. This is due to 
the regression method trying to fit the function to the noise, so this is a direct 
effect of overfitting the function with a to high order polynomial. From figure \eqref{beta OLS}
we see that the beta values start to vary at as low as a fourth order polynomial and that 
for a fifth order polynomial the variance start to become substantially large. This is 
also supported by the MSE plot shown in figure \eqref{MSE and R2 OLS noise}, where we can
see that the MSE starts to gradually increase after the third order polynomial. So we clearly
start to move in to the overfit area. 

We expect that if we set $\lambda = 0$ in our Ridge regression then the result 
will be the same as for OLS. From figur .. and ... we se that when lambda is 
set to zero then the model becoms equal to that for OLS, explenation...


\subsection*{Real terrain data}
\section{Conclusion}
\thispagestyle{plain}
\noindent To summarize our findings, it is clear that the choice of the best regression method depends on the data set used. In scenarios where the data is noise-free, OLS proves to be the most effective regression method. But noise-free data is usually not the case. What we found was that the best regression method for the Franke function with noise given by the normal distribution $\mathcal{N}(0, 0.1)$, was Ridge for higher polynomial degrees, this is due to OLS over-fitting the data to the noise for higher polynomial degrees, while Ridge and LASSO tries to minimise the variance in $\beta$ values to avoid this problem. 
Even though LASSO aims to minimize variance in $\beta$ values, it exhibited the worst performance among the three regression methods. However, it remains uncertain whether Lasso could outperform Ridge and OLS in handling noise due to limitations in iterations. The decision to restrict iterations was influenced by run time constraints, leaving the true capabilities of Lasso in this scenario unknown.
\noindent For our terrain data we surprisingly found that OLS gave the model with the lowest MSE. From what we can see in figure \eqref{OLS 3D figure terrain data} the terrain data looks a little noise, but it may be due to the scaling of the data set that makes OLS the best regression method for the terrain data. 

\noindent The improvement potential for this project is high, due to some grope problems we got started on the project late together as one group. Therefor the structure on GitHub and the cods are quite messy. This is something we will work on improving for the next project. 
% acknowledgements (optional)



%% When it comes to the bibliography I personally generate it using BibLaTeX. (see the link above if you're interested)
%% You're obviously allowed to create the references section however you like.
%% I'll keep it simple here.
\section*{References}  % the asterisk (*) after \section makes the section numbering go away
\begin{itemize}
\item[-]Reference 1
\item[-]Reference 2
\end{itemize}

\newpage
%% if you want to include an appendix, this is how you do it
\clearpage
\onecolumngrid
\appendix
\pagenumbering{alph}

\thispagestyle{plain}
\section{Mean values and variance}\label{sec:appendix_a}
%
\noindent The main regression method used in this report is the ordinary least squares method.
This appensix shows the calculations for some of the equations used to produce the results shown in this report.
%
We have assumed that our data can be described by the continous function 
$f(\boldsymbol{x})$, and an error term $\boldsymbol{\epsilon} ~ N(0, \sigma^{2})$. 
If we approximate the function with the solution derived from a model $\boldsymbol{\tilde{y}} = X\boldsymbol{\beta}$ the data can be described with $\boldsymbol{y} = X\boldsymbol{\beta} + \boldsymbol{\epsilon}$. 
The expectation value 
%
\begin{align*}
    %\hskip\parindent
    \mathbb{E}(\boldsymbol{y}) &= \mathbb{E}(X\boldsymbol{\beta} + \boldsymbol{\epsilon}) \\
    &= \mathbb{E}(X\boldsymbol{\beta}) + \mathbb{E}(\boldsymbol{\epsilon}) && \text{where the expected value $\boldsymbol{\epsilon} = 0$} \\
    \mathbb{E}(y_{i}) &= \sum_{j=0}^{P-1} X_{i,j} \beta_{j} && \text{for the each element} \\
    &= X_{i,*} \beta_{i} && \text{where $_{*}$ replace the sum over index $i$} \\
\end{align*}
%
The variance for the element $y_{i}$ can be found by
\begin{align*}
    \mathbb{V}(y_{i}) &= \mathbb{E} \big[ (y_{i} - \mathbb{E}(y_{i}))^{2} \big] \\
    &= \mathbb{E} (y_{i}^{2}) - (\mathbb{E}(y_{i})^{2}) \\
    &= \mathbb{E} ((X_{i,*} \beta_{i} + \epsilon_{i})^{2}) - (X_{i,*} \beta_{i})^{2} \\
    &= \mathbb{E} ((X_{i,*} \beta_{i})^{2} + 2\epsilon_{i}X_{i,*} \beta_{i} + \epsilon^{2}) - (X_{i,*} \beta_{i})^{2} \\
    &= \mathbb{E} ((X_{i,*} \beta_{i})^{2}) + \mathbb{E} (2\epsilon_{i}X_{i,*} \beta_{i}) + \mathbb{E} (\epsilon^{2}) - (X_{i,*} \beta_{i})^{2} \\
    &= (X_{i,*} \beta_{i})^{2} + \mathbb{E} (\epsilon^{2}) - (X_{i,*} \beta_{i})^{2} \\
    &= \mathbb{E} (\epsilon^{2}) = \sigma^{2} \\
\end{align*}
%
The expression for the optimal parameter 
\begin{align*}
    \boldsymbol{\hat{\beta}} &= (\boldsymbol{X}^{T} \boldsymbol{X})^{-1} \boldsymbol{X}^{T} \boldsymbol{y} \\
\end{align*}
We find the expected value of $\boldsymbol{\hat{\beta}}$
\begin{align*}
    \mathbb{E}(\boldsymbol{\hat{\beta}}) &= \mathbb{E}((\boldsymbol{X}^{T} \boldsymbol{X})^{-1} \boldsymbol{X}^{T} \boldsymbol{y}) \\
    &= (\boldsymbol{X}^{T} \boldsymbol{X})^{-1} \boldsymbol{X}^{T} \mathbb{E}(\boldsymbol{y}) && \text{using that $\boldsymbol{X}$ is a non-stochastic variable} \\
    &= (\boldsymbol{X}^{T} \boldsymbol{X})^{-1} \boldsymbol{X}^{T} \boldsymbol{X} \boldsymbol{\beta} && \text{using $\mathbb{E}(\boldsymbol{y}) = \boldsymbol{X} \boldsymbol{\beta}$} \\
    &= \boldsymbol{\beta} \\
\end{align*}
%
We can find the variance by 
\begin{align*}
    \mathbb{V}(\boldsymbol{\hat{\beta}}) &= \mathbb{E} \big[ (\boldsymbol{\hat{\beta}} - \mathbb{E}(\boldsymbol{\hat{\beta}}))^{2} \big] \\
    &= \mathbb{E} (\boldsymbol{\hat{\beta}} \boldsymbol{\hat{\beta}}^{T}) - \mathbb{E}(\boldsymbol{\hat{\beta}})^{2}  \\
    &= \mathbb{E} (((\boldsymbol{X}^{T} \boldsymbol{X})^{-1} \boldsymbol{X}^{T} \boldsymbol{y}) ((\boldsymbol{X}^{T} \boldsymbol{X})^{-1} \boldsymbol{X}^{T} \boldsymbol{y})^{T}) - \boldsymbol{\hat{\beta}}\boldsymbol{\hat{\beta}}^{T}  \\
    &= \mathbb{E} ((\boldsymbol{X}^{T} \boldsymbol{X})^{-1} \boldsymbol{X}^{T} \boldsymbol{y} \boldsymbol{y}^{T} \boldsymbol{X} (\boldsymbol{X}^{T} \boldsymbol{X})^{-1}) - \boldsymbol{\hat{\beta}}\boldsymbol{\hat{\beta}}^{T}  \\
    &= (\boldsymbol{X}^{T} \boldsymbol{X})^{-1} \boldsymbol{X}^{T} \mathbb{E} (\boldsymbol{y} \boldsymbol{y}^{T}) \boldsymbol{X} (\boldsymbol{X}^{T} \boldsymbol{X})^{-1} - \boldsymbol{\hat{\beta}}\boldsymbol{\hat{\beta}}^{T}  \\
    &= (\boldsymbol{X}^{T} \boldsymbol{X})^{-1} \boldsymbol{X}^{T} (\boldsymbol{X} \boldsymbol{\beta} \boldsymbol{\beta}^{T} \boldsymbol{X}^{T} + \sigma^{2}) \boldsymbol{X} (\boldsymbol{X}^{T} \boldsymbol{X})^{-1} - \boldsymbol{\hat{\beta}}\boldsymbol{\hat{\beta}}^{T}  \\
    &= \boldsymbol{\beta} \boldsymbol{\beta}^{T} + \sigma^{2}((\boldsymbol{X}^{T} \boldsymbol{X})^{-1} \boldsymbol{X}^{T} \boldsymbol{X} (\boldsymbol{X}^{T} \boldsymbol{X})^{-1}) - \boldsymbol{\hat{\beta}}\boldsymbol{\hat{\beta}}^{T}  \\
    &= \sigma^{2}(\boldsymbol{X}^{T} \boldsymbol{X})^{-1} \\
\end{align*}
Knowing the expectation value and the variance of $\boldsymbol{\hat{\beta}}$, we can define a confidence interval for each $\hat{\beta}_{j} \pm std(\hat{\beta}_j)$ for $j=1, 2, \hdots, P-1$.
%
The optimal $\boldsymbol{\hat{\beta}}^{Ridge}$ can be derived from MSE, and is defined as 
\begin{align*}
    \boldsymbol{\hat{\beta}}^{Ridge} &= (\boldsymbol{X}^{T}\boldsymbol{X} + \lambda \boldsymbol{I})^{-1} \boldsymbol{X}^{T} \boldsymbol{y} \\
\end{align*}
The expectation value is then 
\begin{align*}
    \mathbb{E} (\boldsymbol{\hat{\beta}}^{Ridge}) &= \mathbb{E}((\boldsymbol{X}^{T}\boldsymbol{X} + \lambda \boldsymbol{I})^{-1} \boldsymbol{X}^{T} \boldsymbol{y}) \\
    &= (\boldsymbol{X}^{T}\boldsymbol{X} + \lambda \boldsymbol{I})^{-1} \boldsymbol{X}^{T} \mathbb{E}( \boldsymbol{y} ) && \text{since $\boldsymbol{X}$ and $\lambda \boldsymbol{I}$ are non-stochastic variables} \\
    &= (\boldsymbol{X}^{T}\boldsymbol{X} + \lambda \boldsymbol{I})^{-1} \boldsymbol{X}^{T} \boldsymbol{X} \boldsymbol{\beta} && \text{using $\mathbb{E} (\boldsymbol{y})$ from exercise 1} \\
\end{align*}
For $\lambda = 0$ we have $\mathbb{E} (\boldsymbol{\hat{\beta}}^{OLS})$. The variance 
\begin{align*}
    \mathbb{V}(\boldsymbol{\hat{\beta}}^{Ridge}) &= \mathbb{E} (\boldsymbol{\hat{\beta}}_{R} \boldsymbol{\hat{\beta}}_{R}^{T}) - (\mathbb{E} (\boldsymbol{\hat{\beta}}_{R}))^{2} \\
    &= \mathbb{E} (((\boldsymbol{X}^{T}\boldsymbol{X} + \lambda \boldsymbol{I})^{-1} \boldsymbol{X}^{T} \boldsymbol{y}) (((\boldsymbol{X}^{T}\boldsymbol{X} + \lambda \boldsymbol{I})^{-1} \boldsymbol{X}^{T} \boldsymbol{y}))^{T}) - (\mathbb{E} (\boldsymbol{\hat{\beta}}_{R}))^{2} \\
    &= \mathbb{E} ((\boldsymbol{X}^{T}\boldsymbol{X} + \lambda \boldsymbol{I})^{-1} \boldsymbol{X}^{T} \boldsymbol{y} \boldsymbol{y}^{T} \boldsymbol{X} ((\boldsymbol{X}^{T}\boldsymbol{X} + \lambda \boldsymbol{I})^{-1})^{T} ) - (\mathbb{E} (\boldsymbol{\hat{\beta}}_{R}))^{2} \\
    &= (\boldsymbol{X}^{T}\boldsymbol{X} + \lambda \boldsymbol{I})^{-1} \boldsymbol{X}^{T} \mathbb{E} (\boldsymbol{y} \boldsymbol{y}^{T} ) \boldsymbol{X} ((\boldsymbol{X}^{T}\boldsymbol{X} + \lambda \boldsymbol{I})^{-1})^{T} - (\mathbb{E} (\boldsymbol{\hat{\beta}}_{R}))^{2} \\
    &= (\boldsymbol{X}^{T}\boldsymbol{X} + \lambda \boldsymbol{I})^{-1} \boldsymbol{X}^{T} (\boldsymbol{X} \boldsymbol{\beta} \boldsymbol{\beta}^{T} \boldsymbol{X}^{T} + \sigma^{2}) \boldsymbol{X} ((\boldsymbol{X}^{T}\boldsymbol{X} + \lambda \boldsymbol{I})^{-1})^{T} - (\mathbb{E} (\boldsymbol{\hat{\beta}}_{R}))^{2} \\
    &= \sigma^{2}(\boldsymbol{X}^{T}\boldsymbol{X} + \lambda \boldsymbol{I})^{-1} \boldsymbol{X}^{T} \boldsymbol{X} ((\boldsymbol{X}^{T}\boldsymbol{X} + \lambda \boldsymbol{I})^{-1})^{T} + (\mathbb{E} (\boldsymbol{\hat{\beta}}_{R}))^{2} - (\mathbb{E} (\boldsymbol{\hat{\beta}}_{R}))^{2} \\
    &= \sigma^{2}(\boldsymbol{X}^{T}\boldsymbol{X} + \lambda \boldsymbol{I})^{-1} \boldsymbol{X}^{T} \boldsymbol{X} ((\boldsymbol{X}^{T}\boldsymbol{X} + \lambda \boldsymbol{I})^{-1})^{T} \\
\end{align*}


\section{Bias-variance trade-off}\label{sec:appendix_b}
\noindent From equation $\boldsymbol{y} = f(\boldsymbol{x}) + \boldsymbol{\epsilon}$ and the assumption $f(\boldsymbol{x}) \approx \boldsymbol{\tilde{y}} = \boldsymbol{X \beta}$
The expectation value of the mean square error is 
\begin{align*}
	\mathbb{E}((\boldsymbol{y} - \boldsymbol{\tilde{y}})^{2}) &= \mathbb{E}(\boldsymbol{y}^{2}) - 2 \mathbb{E}(\boldsymbol{y} \boldsymbol{\tilde{y}}) \mathbb{E}(\boldsymbol{\tilde{y}}^{2}) \\
	&= \mathbb{E}(f(\boldsymbol{x})^{2}) + \mathbb{E}(\boldsymbol{\epsilon}^{2}) - 2 f(\boldsymbol{x}) \mathbb{E}(\boldsymbol{\tilde{y}}) + \mathbb{V}(\boldsymbol{\tilde{y}}) + \mathbb{E}(\boldsymbol{\tilde{y}})^{2} \\
	&= f(\boldsymbol{x})^{2} - 2 f(\boldsymbol{x}) \mathbb{E}(\boldsymbol{\tilde{y}}) + \mathbb{E}(\boldsymbol{\tilde{y}})^{2} + \mathbb{V}(\boldsymbol{\tilde{y}}) + \mathbb{V}(\boldsymbol{\epsilon}) \\
	&= \mathbb{E}((f(\boldsymbol{x}) - \mathbb{E}(\boldsymbol{\tilde{y}}))^{2}) + \mathbb{V}(\boldsymbol{\tilde{y}}) + \sigma^{2}
\end{align*}

\newpage
\section{Plots}\label{sec:appendix_c}
Here we have collected some interesting plots from the analysis of the Franke function and the terrain data :)
%
\subsection{Franke function}
\begin{figure}[H]
	\centering
	\includegraphics[width=\linewidth]{images/Figure_2.png}
	\caption{A plot showing how a model of different complexities fit the franke function when OLS regession has been used.}
	\label{OLS figure}
\end{figure}
%
\begin{figure}[H]
	\centering
	\includegraphics[width=\linewidth]{images/Figure_6.png}
	\caption{}
	\label{Ridge figure}
\end{figure}

\subsection{Terrain data}
\begin{figure}[H]
	\centering
	\includegraphics[width=\linewidth]{images/Figure_30.png}
	\caption{A 3D plot of the terrain data compared to models created with OLS of complexity 10, 20, 30, 40 and 50}
	\label{OLS 3D figure terrain data}
\end{figure}
%
\begin{figure}[H]
	\centering
	\includegraphics[width=\linewidth]{images/Figure_31.png}
	\caption{A 2D plot of the terrain data compered to models created with OLS of complexity 10, 20, 30, 40 and 50}
	\label{OLS 2D figure terrain data}
\end{figure}
%
\begin{figure}[H]
	\centering
	\includegraphics[width=\linewidth]{images/Figure_30.png}
	\caption{A 3D plot of the terrain data compered to models created with Ridge of complexity 10, 20, 30, 40 and 50 and a $\lambda$ value of $10^{-5}$}
	\label{Ridge 3D figure terrain data}
\end{figure}
%
\begin{figure}[H]
	\centering
	\includegraphics[width=\linewidth]{images/Figure_31.png}
	\caption{A 2D plot of the terrain data compered to models created with Ridge of complexity 10, 20, 30, 40 and 50 and a $\lambda$ value of $10^{-5}$}
	\label{Ridge 2D figure terrain data}
\end{figure}

%% If you want to include figure:
%\includegraphics[scale=1.0]{filename}
%% check https://en.wikibooks.org/wiki/LaTeX/Importing_Graphics if you want to know more

\end{document}

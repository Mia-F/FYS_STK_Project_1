\thispagestyle{plain}

\noindent Since the parameter x and y only had values from 0 to 1 when creating the 
Franke function we did not see it necessary to scale the data.

\noindent From figure \eqref{MSE and R2 OLS} we see that when our data set does not include noise
the MSE for the training data gets lower with increas in complexity, while 
the R2 score gets closer and closer to 1. But when we introduce noise whe see from figure \eqref{MSE and R2 OLS noise}
that for our test data the MSE acctually increase with higher complexity. 

The dataset containing the test data consists of $20\%$ of the original
dataset. This implies that the number of points in the test data is 
significantly smaller than that in the training data. We anticipate 
that by increasing the number of data points, the MSE for the model 
created using the test data will converge towards that of the training
model.

We expect that if we set $\lambda = 0$ in our Ridge regression then the result 
will be the same as for OLS. From figur .. and ... we se that when lambda is 
set to zero then the model becoms equal to that for OLS, explenation...
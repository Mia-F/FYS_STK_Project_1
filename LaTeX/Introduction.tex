\thispagestyle{plain}
\noindent Machine learning is a powerfull tool usfull in many fields of reasarche. 
One illustration of its utility is its application to terrain data analysis.
Through the creation of terrain models from real data of a specific geographic area, 
one can effectively anticipate high-risk avalanche zones\cite{WEN2022103535}, 
potentially leading to life-saving interventions. 
This methods can extends to addressing concerns related to floods, which has become 
a hot topic this past month following the storm Hans. It can also help
in aiding with spatial planning challenges. which is 
usefull in big citys all over the world. It is fair to say machine learning possesses immense potential 
to contribute to the solutions of complex and relevant challenges 
in our modern society, encompassing climate-related issues, urban planning, 
and life-saving endeavors. \newline

\noindent In this report er are going to study three different 
regression methods, ordinary least squares (OLS), 
Ridge and LASSO and see how these method compare to eachother when applied 
to different data sets. 
First we are going to use the Franke function to make dummy data to validate if 
our models works. When plotted in the interval $[0,1]$ this function looks like a mountain and a valley,
which is a perfect starting point when we later want to apply these methods 
on real digital terrain data taken from \url{https://earthexplorer.usgs.gov/}.
To more accuraltly simulate the realism of practical machine learning scenarios, 
we will impose limitations on our datasets. Additionally, we will employ
techniques such as bootstrapping and cross-validation to expand our dataset
size and assess their impact on model validation.

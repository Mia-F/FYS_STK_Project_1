\thispagestyle{plain}
\noindent Machine learning is a powerfull tool usfull in many fields of reasarche. 
One illustration of its utility is its application to terrain data analysis.
Through the creation of terrain models from real data of a specific geographic area, 
one can effectively anticipate high-risk avalanche zones, 
potentially leading to life-saving interventions. 
This methods can extends to addressing concerns related to floods, which has become 
a hot topic this past month (maby write somthing abou Hans). It can also help in aiding with spatial planning challenges. which is 
usefull in big citys all over the world.

\noindent It is fair to say machine learning possesses immense potential 
to contribute to the solutions of complex and relevant challenges 
in our modern society, encompassing climate-related issues, urban planning, 
and life-saving endeavors. \newline

\noindent The aim of this report is to study three different 
regression methods, ordinary least squares (OLS), 
Ridge and LASSO and see how these method compare to eachother when applied 
to different data sets. 
First we are going to look at the Franke function. 
When plotted between 0 and 1 this function looks like a mountain and a valley,
 which is a perfect starting point when we later want to apply these methods 
 on digital terrain data taken from \url{https://earthexplorer.usgs.gov/}.

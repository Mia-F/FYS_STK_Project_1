\thispagestyle{plain}
\section{Mean values and variances calculations}


The main regression method used in this report is the ordinary least squares method.
This appensix shows the calculations for some of the equations used to produce the results shown in this report.
%
We have assumed that our data can be described by the continous function 
$f(\boldsymbol{x})$, and an error term $\boldsymbol{\epsilon} ~ N(0, \sigma^{2})$. 
If we approximate the function with the solution derived from a model $\boldsymbol{\tilde{y}} = X\boldsymbol{\beta}$ the data can be described with $\boldsymbol{y} = X\boldsymbol{\beta} + \boldsymbol{\epsilon}$. 
The expectation value 

\begin{align*}
    %\hskip\parindent
    \mathbb{E}(\boldsymbol{y}) &= \mathbb{E}(X\boldsymbol{\beta} + \boldsymbol{\epsilon}) \\
    &= \mathbb{E}(X\boldsymbol{\beta}) + \mathbb{E}(\boldsymbol{\epsilon}) && \text{where the expected value $\boldsymbol{\epsilon} = 0$} \\
    \mathbb{E}(y_{i}) &= \sum_{j=0}^{P-1} X_{i,j} \beta_{j} && \text{for the each element} \\
    &= X_{i,*} \beta_{i} && \text{where $_{*}$ replace the sum over index $i$} \\
\end{align*}


The variance for the element $y_{i}$ can be found by
\begin{align*}
    \mathbb{V}(y_{i}) &= \mathbb{E} \big[ (y_{i} - \mathbb{E}(y_{i}))^{2} \big] \\
    &= \mathbb{E} (y_{i}^{2}) - (\mathbb{E}(y_{i})^{2}) \\
    &= \mathbb{E} ((X_{i,*} \beta_{i} + \epsilon_{i})^{2}) - (X_{i,*} \beta_{i})^{2} \\
    &= \mathbb{E} ((X_{i,*} \beta_{i})^{2} + 2\epsilon_{i}X_{i,*} \beta_{i} + \epsilon^{2}) - (X_{i,*} \beta_{i})^{2} \\
    &= \mathbb{E} ((X_{i,*} \beta_{i})^{2}) + \mathbb{E} (2\epsilon_{i}X_{i,*} \beta_{i}) + \mathbb{E} (\epsilon^{2}) - (X_{i,*} \beta_{i})^{2} \\
    &= (X_{i,*} \beta_{i})^{2} + \mathbb{E} (\epsilon^{2}) - (X_{i,*} \beta_{i})^{2} \\
    &= \mathbb{E} (\epsilon^{2}) = \sigma^{2} \\
\end{align*}

The expression for the optimal parameter 
\begin{align*}
    \boldsymbol{\hat{\beta}} &= (\boldsymbol{X}^{T} \boldsymbol{X})^{-1} \boldsymbol{X}^{T} \boldsymbol{y} \\
\end{align*}
We find the expected value of $\boldsymbol{\hat{\beta}}$
\begin{align*}
    \mathbb{E}(\boldsymbol{\hat{\beta}}) &= \mathbb{E}((\boldsymbol{X}^{T} \boldsymbol{X})^{-1} \boldsymbol{X}^{T} \boldsymbol{y}) \\
    &= (\boldsymbol{X}^{T} \boldsymbol{X})^{-1} \boldsymbol{X}^{T} \mathbb{E}(\boldsymbol{y}) && \text{using that $\boldsymbol{X}$ is a non-stochastic variable} \\
    &= (\boldsymbol{X}^{T} \boldsymbol{X})^{-1} \boldsymbol{X}^{T} \boldsymbol{X} \boldsymbol{\beta} && \text{using $\mathbb{E}(\boldsymbol{y}) = \boldsymbol{X} \boldsymbol{\beta}$} \\
    &= \boldsymbol{\beta} \\
\end{align*}
we can find the variance by 
\begin{align*}
    \mathbb{V}(\boldsymbol{\hat{\beta}}) &= \mathbb{E} \big[ (\boldsymbol{\hat{\beta}} - \mathbb{E}(\boldsymbol{\hat{\beta}}))^{2} \big] \\
    &= \mathbb{E} (\boldsymbol{\hat{\beta}} \boldsymbol{\hat{\beta}}^{T}) - \mathbb{E}(\boldsymbol{\hat{\beta}})^{2}  \\
    &= \mathbb{E} (((\boldsymbol{X}^{T} \boldsymbol{X})^{-1} \boldsymbol{X}^{T} \boldsymbol{y}) ((\boldsymbol{X}^{T} \boldsymbol{X})^{-1} \boldsymbol{X}^{T} \boldsymbol{y})^{T}) - \boldsymbol{\hat{\beta}}\boldsymbol{\hat{\beta}}^{T}  \\
    &= \mathbb{E} ((\boldsymbol{X}^{T} \boldsymbol{X})^{-1} \boldsymbol{X}^{T} \boldsymbol{y} \boldsymbol{y}^{T} \boldsymbol{X} (\boldsymbol{X}^{T} \boldsymbol{X})^{-1}) - \boldsymbol{\hat{\beta}}\boldsymbol{\hat{\beta}}^{T}  \\
    &= (\boldsymbol{X}^{T} \boldsymbol{X})^{-1} \boldsymbol{X}^{T} \mathbb{E} (\boldsymbol{y} \boldsymbol{y}^{T}) \boldsymbol{X} (\boldsymbol{X}^{T} \boldsymbol{X})^{-1} - \boldsymbol{\hat{\beta}}\boldsymbol{\hat{\beta}}^{T}  \\
    &= (\boldsymbol{X}^{T} \boldsymbol{X})^{-1} \boldsymbol{X}^{T} (\boldsymbol{X} \boldsymbol{\beta} \boldsymbol{\beta}^{T} \boldsymbol{X}^{T} + \sigma^{2}) \boldsymbol{X} (\boldsymbol{X}^{T} \boldsymbol{X})^{-1} - \boldsymbol{\hat{\beta}}\boldsymbol{\hat{\beta}}^{T}  \\
    &= \boldsymbol{\beta} \boldsymbol{\beta}^{T} + \sigma^{2}((\boldsymbol{X}^{T} \boldsymbol{X})^{-1} \boldsymbol{X}^{T} \boldsymbol{X} (\boldsymbol{X}^{T} \boldsymbol{X})^{-1}) - \boldsymbol{\hat{\beta}}\boldsymbol{\hat{\beta}}^{T}  \\
    &= \sigma^{2}(\boldsymbol{X}^{T} \boldsymbol{X})^{-1} \\
\end{align*}
Knowing the expectation value and the variance of $\boldsymbol{\hat{\beta}}$, we can define a confidence intervall for each $\hat{\beta}_{j} \pm std(\hat{\beta}_j)$ for $j=1, 2, \hdots, P-1$.

The optimal $\boldsymbol{\hat{\beta}}^{Ridge}$ can be derived from MSE, and is defined as 
\begin{align*}
    \boldsymbol{\hat{\beta}}^{Ridge} &= (\boldsymbol{X}^{T}\boldsymbol{X} + \lambda \boldsymbol{I})^{-1} \boldsymbol{X}^{T} \boldsymbol{y} \\
\end{align*}
The expectation value is then 
\begin{align*}
    \mathbb{E} (\boldsymbol{\hat{\beta}}^{Ridge}) &= \mathbb{E}((\boldsymbol{X}^{T}\boldsymbol{X} + \lambda \boldsymbol{I})^{-1} \boldsymbol{X}^{T} \boldsymbol{y}) \\
    &= (\boldsymbol{X}^{T}\boldsymbol{X} + \lambda \boldsymbol{I})^{-1} \boldsymbol{X}^{T} \mathbb{E}( \boldsymbol{y} ) && \text{since $\boldsymbol{X}$ and $\lambda \boldsymbol{I}$ are non-stochastic variables} \\
    &= (\boldsymbol{X}^{T}\boldsymbol{X} + \lambda \boldsymbol{I})^{-1} \boldsymbol{X}^{T} \boldsymbol{X} \boldsymbol{\beta} && \text{using $\mathbb{E} (\boldsymbol{y})$ from exercise 1} \\
\end{align*}
For $\lambda = 0$ we have $\mathbb{E} (\boldsymbol{\hat{\beta}}^{OLS})$. The variance 
\begin{align*}
    \mathbb{V}(\boldsymbol{\hat{\beta}}^{Ridge}) &= \mathbb{E} (\boldsymbol{\hat{\beta}}_{R} \boldsymbol{\hat{\beta}}_{R}^{T}) - (\mathbb{E} (\boldsymbol{\hat{\beta}}_{R}))^{2} \\
    &= \mathbb{E} (((\boldsymbol{X}^{T}\boldsymbol{X} + \lambda \boldsymbol{I})^{-1} \boldsymbol{X}^{T} \boldsymbol{y}) (((\boldsymbol{X}^{T}\boldsymbol{X} + \lambda \boldsymbol{I})^{-1} \boldsymbol{X}^{T} \boldsymbol{y}))^{T}) - (\mathbb{E} (\boldsymbol{\hat{\beta}}_{R}))^{2} \\
    &= \mathbb{E} ((\boldsymbol{X}^{T}\boldsymbol{X} + \lambda \boldsymbol{I})^{-1} \boldsymbol{X}^{T} \boldsymbol{y} \boldsymbol{y}^{T} \boldsymbol{X} ((\boldsymbol{X}^{T}\boldsymbol{X} + \lambda \boldsymbol{I})^{-1})^{T} ) - (\mathbb{E} (\boldsymbol{\hat{\beta}}_{R}))^{2} \\
    &= (\boldsymbol{X}^{T}\boldsymbol{X} + \lambda \boldsymbol{I})^{-1} \boldsymbol{X}^{T} \mathbb{E} (\boldsymbol{y} \boldsymbol{y}^{T} ) \boldsymbol{X} ((\boldsymbol{X}^{T}\boldsymbol{X} + \lambda \boldsymbol{I})^{-1})^{T} - (\mathbb{E} (\boldsymbol{\hat{\beta}}_{R}))^{2} \\
    &= (\boldsymbol{X}^{T}\boldsymbol{X} + \lambda \boldsymbol{I})^{-1} \boldsymbol{X}^{T} (\boldsymbol{X} \boldsymbol{\beta} \boldsymbol{\beta}^{T} \boldsymbol{X}^{T} + \sigma^{2}) \boldsymbol{X} ((\boldsymbol{X}^{T}\boldsymbol{X} + \lambda \boldsymbol{I})^{-1})^{T} - (\mathbb{E} (\boldsymbol{\hat{\beta}}_{R}))^{2} \\
    &= \sigma^{2}(\boldsymbol{X}^{T}\boldsymbol{X} + \lambda \boldsymbol{I})^{-1} \boldsymbol{X}^{T} \boldsymbol{X} ((\boldsymbol{X}^{T}\boldsymbol{X} + \lambda \boldsymbol{I})^{-1})^{T} + (\mathbb{E} (\boldsymbol{\hat{\beta}}_{R}))^{2} - (\mathbb{E} (\boldsymbol{\hat{\beta}}_{R}))^{2} \\
    &= \sigma^{2}(\boldsymbol{X}^{T}\boldsymbol{X} + \lambda \boldsymbol{I})^{-1} \boldsymbol{X}^{T} \boldsymbol{X} ((\boldsymbol{X}^{T}\boldsymbol{X} + \lambda \boldsymbol{I})^{-1})^{T} \\
\end{align*}


\section{Bias-variance trade-off}
From equation $\boldsymbol{y} = f(\boldsymbol{x}) + \boldsymbol{\epsilon}$ and the assumption $f(\boldsymbol{x}) \approx \boldsymbol{\tilde{y}} = \boldsymbol{X \beta}$
The expectation value of the mean square error is 
\begin{align*}\label{eq:bias_variance}
	\mathbb{E}((\boldsymbol{y} - \boldsymbol{\tilde{y}})^{2}) &= \mathbb{E}(\boldsymbol{y}^{2}) - 2 \mathbb{E}(\boldsymbol{y} \boldsymbol{\tilde{y}}) \mathbb{E}(\boldsymbol{\tilde{y}}^{2}) \\
	&= \mathbb{E}(f(\boldsymbol{x})^{2}) + \mathbb{E}(\boldsymbol{\epsilon}^{2}) - 2 f(\boldsymbol{x}) \mathbb{E}(\boldsymbol{\tilde{y}}) + \mathbb{V}(\boldsymbol{\tilde{y}}) + \mathbb{E}(\boldsymbol{\tilde{y}})^{2} \\
	&= f(\boldsymbol{x})^{2} - 2 f(\boldsymbol{x}) \mathbb{E}(\boldsymbol{\tilde{y}}) + \mathbb{E}(\boldsymbol{\tilde{y}})^{2} + \mathbb{V}(\boldsymbol{\tilde{y}}) + \mathbb{V}(\boldsymbol{\epsilon}) \\
	&= \mathbb{E}((f(\boldsymbol{x}) - \mathbb{E}(\boldsymbol{\tilde{y}}))^{2}) + \mathbb{V}(\boldsymbol{\tilde{y}}) + \sigma^{2}
\end{align*}

